\pdfminorversion=4
\documentclass[aspectratio=169]{beamer}

\mode<presentation>
{
  \usetheme{default}
  \usecolortheme{default}
  \usefonttheme{default}
  \setbeamertemplate{navigation symbols}{}
  \setbeamertemplate{caption}[numbered]
  \setbeamertemplate{footline}[frame number]  % or "page number"
  \setbeamercolor{frametitle}{fg=white}
  \setbeamercolor{footline}{fg=black}
} 

\usepackage[english]{babel}
\usepackage[utf8x]{inputenc}
\usepackage{tikz}
\usepackage{courier}
\usepackage{array}
\usepackage{bold-extra}
\usepackage{minted}
\usepackage[thicklines]{cancel}
\usepackage{fancyvrb}

\xdefinecolor{dianablue}{rgb}{0.18,0.24,0.31}
\xdefinecolor{darkblue}{rgb}{0.1,0.1,0.7}
\xdefinecolor{darkgreen}{rgb}{0,0.5,0}
\xdefinecolor{darkgrey}{rgb}{0.35,0.35,0.35}
\xdefinecolor{darkorange}{rgb}{0.8,0.5,0}
\xdefinecolor{darkred}{rgb}{0.7,0,0}
\definecolor{darkgreen}{rgb}{0,0.6,0}
\definecolor{mauve}{rgb}{0.58,0,0.82}

\title[2021-12-06-uvirginia-hep-languages]{Programming Languages, Toolkits, and Communities \\ in Particle Physics Data Analysis}
\author{Jim Pivarski}
\institute{Princeton University -- IRIS-HEP}
\date{December 6, 2021}

\usetikzlibrary{shapes.callouts}

\begin{document}

\logo{\pgfputat{\pgfxy(0.11, 7.4)}{\pgfbox[right,base]{\tikz{\filldraw[fill=dianablue, draw=none] (0 cm, 0 cm) rectangle (50 cm, 1 cm);}\mbox{\hspace{-8 cm}\includegraphics[height=1 cm]{princeton-logo-long.png}\hspace{0.1 cm}\raisebox{0.1 cm}{\includegraphics[height=0.8 cm]{iris-hep-logo-long.png}}\hspace{0.1 cm}}}}}

\begin{frame}
  \titlepage
\end{frame}

\logo{\pgfputat{\pgfxy(0.11, 7.4)}{\pgfbox[right,base]{\tikz{\filldraw[fill=dianablue, draw=none] (0 cm, 0 cm) rectangle (50 cm, 1 cm);}\mbox{\hspace{-8 cm}\includegraphics[height=1 cm]{princeton-logo.png}\hspace{0.1 cm}\raisebox{0.1 cm}{\includegraphics[height=0.8 cm]{iris-hep-logo.png}}\hspace{0.1 cm}}}}}

% Uncomment these lines for an automatically generated outline.
%\begin{frame}{Outline}
%  \tableofcontents
%\end{frame}

% START START START START START START START START START START START START START

\begin{frame}{First slide}
\vspace{0.5 cm}
HERE
\end{frame}

\end{document}

%% Experimental particle physics is an intensely computational field of science. In fact, particle physicists were arguably the first non-secret (non-cryptography) users of digital computers, and have been pushing the boundaries of pattern recognition and throughput ever since. For decades, our unique needs justified custom software at all levels of the stack, maintained "in-house" by physicists, but the situation changed in the 21st century. Machine learning and analysis of web-scale datasets (i.e. "Big Data") has become an industry on its own, under the catch-all name "data science." Physicists are responding by adopting data science toolsets and methodologies, integrating them with traditional physics software, though the process is ongoing and differs in degree across physics groups. 

%% This talk will present a big picture of how experimental particle physicists have used data analysis software in the past 75 years, how our needs have dictated a choice of programming languages and toolkits, and how those choices are changing. We'll see how pattern recognition evolved from semi-automated to algorithmic to machine learning, how programming languages transitioned from Fortran to C++ to include a significant mix of Python, and how software was organized from site-custom solutions to standard packages like CERNLIB and ROOT to also include a mix of data science tools. Finally, these choices are not purely technical: communities form around software tools, and integrating toolsets integrates physicists with the larger world.

% "neural" and (publication_info.cnum:C85-06-25 or publication_info.cnum:C87-02-02.2 or publication_info.cnum:C89-04-10 or publication_info.cnum:C90-04-09 or publication_info.cnum:C91-03-11 or publication_info.cnum:C92-09-21 or publication_info.cnum:C94-04-21 or publication_info.cnum:C95-09-18 or publication_info.cnum:C97-04-07 or publication_info.cnum:C98-08-31 or publication_info.cnum:C00-02-07 or publication_info.cnum:C01-09-03.1 or publication_info.cnum:C03-03-24.1 or publication_info.cnum:C04-09-27 or publication_info.cnum:C06-02-13 or publication_info.cnum:C07-09-02.1 or publication_info.cnum:C09-03-21 or publication_info.cnum:C10-10-18.4 or publication_info.cnum:C12-05-21.3 or publication_info.cnum:C13-10-14.1 or publication_info.cnum:C15-04-13 or publication_info.cnum:C16-10-14 or publication_info.cnum:C18-07-09.6 or publication_info.cnum:C19-11-04 or publication_info.cnum:C21-05-17.1)

% curl -s 'https://inspirehep.net/api/literature?sort=mostrecent&size=25&page=1&q=%28publication_info.cnum%3AC85-06-25%20or%20publication_info.cnum%3AC87-02-02.2%20or%20publication_info.cnum%3AC89-04-10%20or%20publication_info.cnum%3AC90-04-09%20or%20publication_info.cnum%3AC91-03-11%20or%20publication_info.cnum%3AC92-09-21%20or%20publication_info.cnum%3AC94-04-21%20or%20publication_info.cnum%3AC95-09-18%20or%20publication_info.cnum%3AC97-04-07%20or%20publication_info.cnum%3AC98-08-31%20or%20publication_info.cnum%3AC00-02-07%20or%20publication_info.cnum%3AC01-09-03.1%20or%20publication_info.cnum%3AC03-03-24.1%20or%20publication_info.cnum%3AC04-09-27%20or%20publication_info.cnum%3AC06-02-13%20or%20publication_info.cnum%3AC07-09-02.1%20or%20publication_info.cnum%3AC09-03-21%20or%20publication_info.cnum%3AC10-10-18.4%20or%20publication_info.cnum%3AC12-05-21.3%20or%20publication_info.cnum%3AC13-10-14.1%20or%20publication_info.cnum%3AC15-04-13%20or%20publication_info.cnum%3AC16-10-14%20or%20publication_info.cnum%3AC18-07-09.6%20or%20publication_info.cnum%3AC19-11-04%20or%20publication_info.cnum%3AC21-05-17.1%29' 2>&1 > all-chep-papers.json

%% {
%%     "C85-06-25": 1,
%%     "C87-02-02.2": 2,
%%     "C89-04-10": 3,
%%     "C90-04-09": 4,
%%     "C91-03-11": 5,
%%     "C92-09-21": 6,
%%     "C94-04-21": 7,
%%     "C95-09-18": 8,
%%     "C97-04-07": 9,
%%     "C98-08-31": 10,
%%     "C00-02-07": 11,
%%     "C01-09-03.1": 12,
%%     "C03-03-24.1": 13,
%%     "C04-09-27": 14,
%%     "C06-02-13": 15,
%%     "C07-09-02.1": 16,
%%     "C09-03-21": 17,
%%     "C10-10-18.4": 18,
%%     "C12-05-21.3": 19,
%%     "C13-10-14.1": 20,
%%     "C15-04-13": 21,
%%     "C16-10-14": 22,
%%     "C18-07-09.6": 23,
%%     "C19-11-04": 24,
%%     "C21-05-17.1": 25,
%% }
